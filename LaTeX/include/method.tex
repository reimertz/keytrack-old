\chapter{Method}
The project was divided into three different phases. The literature study, the developing phase and the test/polishing phase.

The first phase was where all the information needed to make the project possible was gathered.
The second phase was the phase where the prototype was created and it's functionality implemented.
Finally, the last phase was used to make the finishing touches on the prototype to make sure it
was working as expected.

\section{Literature review}
During the literature review there were several solutions to consider. Ultrasound, IR and an optic
solution. These solutions were documented in databases that Chalmers provides for its students.
Information was gathered from various websites, articles and reports inside and outside of the
database.

Along with finding a proper solutions, proper hardware was required for the project.\\
When gathering information from reports, articles and from other sources, a lot of hardware was
presented. This made it easier to chose hardware because of the natural combination. The hardware
information itself was gathered from it's creators website, since some of the articles and reports
didn't provide enough information for what the project needed.

\section{Developing phase}
\subsection{Hardware}
The first small prototype was made using a breadboard and an Arduino Uno. The prototype
include four phototransistors and a single IR-LED. This first edition was capable of interpreting
sideways mouvements and left-mouse-clicks. Because of the limited space on the breadboard
the next version of the prototype had to be made using some other base on which to put the 
hardware.\\
Stripboards were cut into circa 14 centimeter long pieces and put together to make a square
frame on which the hardware was fastened using wire-wrapping.

Wire-wrapping is an alternative to and not as skill-demanding as soldering.

\subsection{Software}

\subsection{Design}
\subsubsection{Frame design}
This design of this project is based on a “keyboard name” keyboard. The frame is 
entirely based on the measures of the keyboard. This includes the height and length, 
since it’s important that the frame is not in the way when operating the keyboard. 
With this in mind the prototype was given the measurements “lol”, “lol”, “lol”. The 
frame is connected with a lot of cables to a arduino card at the other end. Even 
though these cables will hurt the design and the prototype’s adaptability, this project 
will not treat the matter due to limited time. From the arduino card goes a 
USB cable (mini to 2.0 USB)  into the computer.

\subsubsection{Mouse movements and clicks}
 Since the prototype was created to replace a normal mouse, this chapter is showing 
 how the different commands are translated from the mouse to the IR-frame. 
 Since the mouse we choose as reference has 3 buttons, the prototype won’t go any 
 further than that. The scroll-click is also excluded.

The card Arduino Due that is used in this project has already a software library 
with mouse commands. This made it easy to locate and translate the mouse command in to
what ever command desired.

The different mouse commands were then translated to the IR-frame. This proved to be 
difficult since it was not obvious what hand gestures worked best with what command. 

\section{Test/polishing phase}
N/A